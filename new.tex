\documentclass {beamer}

\usepackage[spanish]{babel}
\usepackage[utf8]{inputenc}
\usepackage{graphicx}

\newcommand{\PI}{{$\pi$}}
\title{Número PI}
\author{Adonis Miguel Martín Flores}
\institute{Fac. Mat.}
\date{24 de abril de 2014}
\usetheme{Madrid}

\definecolor{MiVioleta}{RGB}{122, 59, 122}
\definecolor{MiAzul}{RGB}{0,88,147}
\definecolor{MiGris}{RGB}{56,61,66}
\setbeamercolor*{palette primary}{use=structure, fg=white, bg=MiVioleta}
\setbeamercolor*{palette secundary}{use=structure, fg=white, bg=MiAzul}
\setbeamercolor*{palette tertiary}{use=structure, fg=white, bg=MiGris}
\begin{document}


\begin{frame}
\titlepage
\end{frame}
\begin{frame}
\frametitle{Indice}
\tableofcontents
\end{frame}
\section {Introducción}
\begin{frame}
\frametitle{Introduccion}
\PI es la relación entre la longitud de una circunferencia y su diámetro, en geometría euclidiana. Es un número irracional y una de las constantes matemáticas más importantes. Se emplea frecuentemente en matemáticas, física e ingeniería.
\end{frame}
\section{Historia del cálculo del número \PI}
\begin{frame}
El valor aproximado de \PI en las antiguas culturas se remonta a la época del escriba egipcio Ahmes en el año 1800 a. C., descrito en el papiro Rhind,5 donde se emplea un valor aproximado de \PI afirmando que el área de un círculo es similar a la de un cuadrado cuyo lado es igual al diámetro del círculo disminuido en $1/9$; es decir, igual a $8/9$ del diámetro. En notación moderna:

$ S = \PI r^2 \approxeq (frac{8}{9} * d)^2 = frac{64}{81}d^2 =frac{64}{81}(4r^2)$

$ \PI \approxeq frac{256}{81} = 3,16049...$
\end{frame}
\section{Características matemáticas}
\frametitle{Características}
\begin{frame}
Euclides fue el primero en demostrar que la relación entre una circunferencia y su diámetro es una cantidad constante. No obstante existen diversas definiciones del numero \PI , pero la más común es:
\begin{itemize}
\item \PI es la razón entre la longitud de cualquier circunferencia y su diámetro.
Además \PI es:
\item El área de un círculo unitario (de radio tiene longitud 1, en el plano geométrico usual o plano euclídeo).
\item El menos número real x positivo tal que $sin(x) = 0.$
\end{itemize}
\end{frame}
\section{Bibliografía}
\begin{frame}
Bibliografía
\begin{verbatim}

es.wikipedia.org/wiki/Número\_n
\end{verbatim} 
\end{frame}
\end{document}